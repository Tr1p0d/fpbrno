\documentclass[11pt]{beamer}

\usepackage[utf8]{inputenc}
\usepackage[T1]{fontenc}
\usepackage{amsmath}
\usepackage{amsfonts}
\usepackage{amssymb}
\usepackage{graphicx}
\usepackage{tikz}
\usepackage{filecontents}
\usepackage{pgfplots}
\usepackage{listings}

\usetheme{default}

\lstset{
	frame=none,
	xleftmargin=2pt,
	stepnumber=1,
	numbers=left,
	numbersep=5pt,
	numberstyle=\ttfamily\tiny\color[gray]{0.3},
	belowcaptionskip=\bigskipamount,
	captionpos=b,
	escapeinside={*'}{'*},
	language=haskell,
	tabsize=2,
	emphstyle={\bf},
	commentstyle=\it,
	stringstyle=\mdseries\rmfamily,
	showspaces=false,
	keywordstyle=\bfseries\rmfamily,
	columns=flexible,
	basicstyle=\footnotesize\ttfamily,
	showstringspaces=false,
	morecomment=[l]\%,
}

\author{Marek Kido\v{n}}
\title{Purely functional programming}
	\subtitle{There is no turning back}
	%\logo{}
	%\institute{}
\date{\today}
	%\subject{}
	%\setbeamercovered{transparent}
	%\setbeamertemplate{navigation symbols}{}

\begin{document}	
	\begin{frame}[plain]
		\titlepage
%		\addtocounter{framenumber}{-1}
	\end{frame}

	\begin{frame}
		\frametitle{What is purely functional programming}
			\begin{block}{Functional}
				\begin{itemize}[<+->]
					\item Style of  building the structure of computer programs
					\item Declarative: the job is done using expressions(declarations), not statements
					\item Build up functions from expressions
					\item Function output depends on inputs only
				\end{itemize}
			\end{block}
		
			\begin{block}{Purely functional}
				\begin{itemize}[<+->]
					\item Controversy: a lot of things to a lot of people
					\item How do you handle computational side effects?
					\item Treatment of computational effects is the key
				\end{itemize}
			\end{block}
	\end{frame}

	\begin{frame}
		\frametitle{Types: Motivation}
			\begin{itemize}[<+->]
				\item Mars Climate Orbiter disintegrated during trajectory correction in 1999.
				\item Cause: Software provided by Lockheed Martin calculated in pound-seconds. NASA expected newton-seconds. Cost: \$125 million.
				\item Ariane 5 was destructed by engineers shortly after launch.
				\item Cause: Computers tried to cram 64bit number into 16bit space. Cost: \$500 million worth of the carried satellite.
				\item Some dude's website kicked a bucket.
				\item Cause: An obscure function expected a capitalized string but received a lowercase one. Cost: a really bad day for some dude and his employer.
				\item Every day a lot of money is wasted on bugs, their consequences and their fixing (which often introduces new bugs...).
			\end{itemize}
	\end{frame}

	\begin{frame}
		\frametitle{Types: What are they?}
		\begin{itemize}[<+->]
			\item Types and static type systems are programmers best friend!
			\item Controversy: a lot of things to a lot of people. 
			\item Definition: Syntactic? Representational? Behavioral? Value space definition?
			\item Common sense: Type is uniquely defined by the set of its values. Think of integers, characters, tuples, lists, etc...
		\end{itemize}
	\end{frame}

	\begin{frame}
		\frametitle{Types: The epic fail...}
			\begin{itemize}[<+->]
				\item How do you define a new type in most programming languages?
				\item Are data structures or objects the solution?
				\item Question: How do you represent the Logic = \{True, False, Unknown, Uninitialized, \ldots \} type used in hardware related applications?
				\item Answer: You can't really... (or use language with such type built in)
				\item Common practice is to map type semantics into a structure of existing types. Are their sets of values isomorphic?
			\end{itemize}
	\end{frame}

\begin{frame}[fragile]
\frametitle{Types: Example 1}
\begin{lstlisting}
#define TRUE 0
#define FALSE 1
#define UNKNOWN 2
#define UNINITIALIZED 3
typedef unsigned int STD_LOGIC;
void launchViciousRocket(STD_LOGIC);

int main(void) {
	STD_LOGIC signalValue;

	signalValue = UNKNOWN;
	signalValue = 0xdeadbeef;

	launchViciousRocket(signalValue);
	
	return 0;
}
\end{lstlisting}

\begin{itemize}[<+->]
	\item Question: will be the vicious rocket launched?
	\item Answer: Who knows...
\end{itemize}
\end{frame}

	\begin{frame}
		\frametitle{Types: The purely functional approach}
			\begin{itemize}[<+->]
				\item You can create new types as labels for their sets of values.
				\item An instance of a type is created by a creating to a certain value in its set.
				\item Pattern matching: as opposed to creating, you have a way to dismantle it.
			\end{itemize}
	\end{frame}

\begin{frame}[fragile]
	\frametitle{Types: Example 2}
\begin{lstlisting}[language=haskell]
launchViciousRocket 
    :: StdLogic 
    -> PossiblyKillMillionsOfPeople

data StdLogic
    = TRUE
    | FALSE
    | Unknown
    | Uninitialized
	
main = 
    let signalValue = 0xdeadbeef 
    in launchViciousRocket(signalValue)		
\end{lstlisting}
\begin{itemize}[<+->]
	\item Question: will be the vicious rocket launched?
	\item Answer: No, such program will never compile.
\end{itemize}		
\end{frame}

\begin{frame}
	\frametitle{Types: The lesson}
	\begin{itemize}[<+->]
		\item Computer program's semantics can be implemented in types.
		\item Type correctness can be checked at compile time!
		\item Sadly you can't create new types in most programming languages.
		\item You rely on run-time checks instead of compile-time type checker.
	\end{itemize}	
\end{frame}

\begin{frame}
	\frametitle{The Vision case study}
	\begin{block}{Task}
		\begin{itemize}[<+->]
			\item Write an automated quality control system for Hyundai.
			\item You have less than 2 months for it.
			\item Two people project.
			\item Customer has to sign an acceptation protocol.
		\end{itemize}
	\end{block}
	\begin{block}{Result}
		\begin{itemize}[<+->]
			\item 10000 lines of code in Haskell
			\item Few hundreds lines of C++/openCV using functional techniques.
			\item Not a single unit test
			\item Some functional tests
			\item Acceptation protocol signed
			\item Still waiting for first bug report
		\end{itemize}
	\end{block}
\end{frame}

\begin{frame}
	\frametitle{PFP: Programmers mindset}		
	\begin{itemize}[<+->]
		\item Problem: People do make mistakes. 
		\item Solution: Computers do not. Transfer as much work on a computer as possible.
		\item Problem: People hate to writing boilerplate code. It is repetitious, boring and error prone.
		\item Solution: Use as many abstract constructs as possible. Do not write boilerplate code. Good compiler can generate it for you.
		\item Problem?: People are lazy.
		\item Solution: Be lazy to type. \textit{Do not be lazy to think}. Encode semantics in types whenever it is beneficial. Let the compiler do your work.
	\end{itemize}
\end{frame}

\begin{frame} 
	\frametitle{PFP: What does it bring to a company}		
	\begin{itemize}[<+->]
		\item Imagine you will do the next project in Haskell...
		\item What are the consequences?
		\item You get smart and creative people by definition(previous slide).
		\item What is their motivation?
		\item Your software is more likely to be less buggy
	\end{itemize}
\end{frame}

\begin{frame}
	\frametitle{PFP: There is no turning back}
	\begin{itemize}[<+->]
		\item Do not fear the purely functional platforms. They are often the \textit{final frontier} of programming techniques.
		\item Imagine you decide to learn purely functional programming incorporating strong static typing.
		\item Problem: \textit{there is no turning back}
		\item Cause: You get lazy beyond belief
		\item Solution: its not really a problem is it?
	\end{itemize}
\end{frame}
\end{document}